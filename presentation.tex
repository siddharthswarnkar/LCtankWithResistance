\documentclass[11pt]{article}
\usepackage{graphicx}
\usepackage{amsmath}
\usepackage{amssymb}

\title{LC Tank With Resistance}
\author{Siddharth Swarnkar \\ 
Indian Institute of Technology, Bombay}

\begin{document}
\maketitle

\begin{abstract}
This paper presents the study of the behaviour of voltage and current in electric circuit consisting of  Resistance(R), Capacitor(C) and Inductor(L).\\
The behaviour is studied under different arrangement of RLC including series and parallel arrangements. The step response of the system under constant voltage source is also presented.
\end{abstract}

\section{Introduction}
An RLC circuit is an electrical circuit consisting of a resistor (R), an inductor (L), and a capacitor (C), connected in series or in parallel.\\
The circuit forms a harmonic oscillator for current, and resonates in a similar way as an LC circuit. Introducing the resistor increases the decay of these oscillations, which is also known as damping. 
\section{Source-Free Series RLC circuit}
The circuit is being excited by the energy initially stored in the capacitor and inductor.The energy is represented by the initial capacitor voltage and initial inductor current . Thus, at t=0,\\
\begin{align}
v(0) &= \frac{1}{C}\int_{-\infty}^{0}idt = V_o \label{1} \\
i(0) &= I_o \label{2}
\end{align}
Applying KVL around the loop we get \\
\begin{align}
Ri+L\frac{di}{dt}+\frac{1}{C}\int_{-\infty}^{0}i(\tau)d\tau = 0 \label{3}
\end{align}
differentiating with respect to t, we get \\
\begin{align}
\frac{d^2i}{dt^2} + \frac{R}{L}\frac{di}{dt}+\frac{i}{LC}=0 \label{4}
\end{align}

We get initial condition of current(i) using \ref{2} and initial condition of $\frac{di}{dt}$ can be determined using \ref{1} and \ref{3}; that is,\\
$$ Ri(0)+L\frac{di(0)}{dt}+V_o=0 $$ \\or\\
\begin{align}
\frac{di(0)}{dt} = \frac{-(RI_o+V_o)}{L}
\end{align}

These initial conditions are used to solve \ref{4} numerically using scipy odeint function.

\section{Source-Free parallel RLC circuit}
Parallel RLC circuits find many practical applications, notably in communications networks and filter designs.
Assume initial inductor current $I_o$ and initial capacitor voltage $V_o$ 
\begin{align}
i(0) &= I_o = \frac{1}{L}\int_{\infty}^{0}v(t)dt \label{5}\\
v(0) &= V_o \label{6}
\end{align}
Applying KVL we get,
 \begin{align}
 \frac{v}{R} + \frac{1}{L}\int_{-\infty}^tv(\tau)d\tau + C\frac{dv}{dt} \label{7}
 \end{align}
 Taking the derivative with respect to t and dividing by C results in
 \begin{align}
 \frac{d^2v}{dt^2} + \frac{1}{RC}\frac{dv}{dt} + \frac{1}{LC}v =0 \label{8}
 \end{align}
 For initial condition we need to find $v(0)$ and $\frac{dv(0)}{dt}$. The first term is known is from \ref{6}.
 We find second term from \ref{5} and \ref{7}, as
 \begin{align}
 \frac{dv(0)}{dt} = \frac{-(V_o+RI_o)}{RC}
 \end{align}
 These initial conditions are used to solve \ref{8} numerically using scipy odeint function.

\section{Step Response of Series RLC circuit}
The step response is obtained by the sudden application of a dc source. Applying KVL around the loop for $t>0$,
\begin{align}
 &L\frac{di}{dt} + Ri + v = V_s \label{9} \\
 &i = L\frac{dv}{dt}
\end{align}
 Substituting for i in equation \ref{9}  and rearranging terms,
 $$ \frac{d^2v}{dt^2} + \frac{R}{L}\frac{dv}{dt} + \frac{v}{LC} =  \frac{V_s}{LC} $$
 Initial condition for this case are $v(0) = 0$ and $\frac{dv(0)}{dt} = 0 $
\end{document}
